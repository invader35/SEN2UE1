\documentclass[10pt,a4paper]{scrartcl}
\usepackage[utf8x]{inputenc}
\usepackage[german]{babel}
\usepackage{ucs}
\usepackage{amsmath}
\usepackage{amsfonts}
\usepackage{amssymb}
\usepackage{graphicx}
\usepackage{color}
\usepackage{listings}
\usepackage{hyperref}
\author{Schett Matthias}
\title{SEN-Übung 2.1}
\begin{document}
\definecolor{gray}{rgb}{0.9,0.9,0.9}
\lstset{language=[Visual]Basic, morekeywords={param, local}}
\maketitle

\newpage
\section{Aufgabe 1}

\subsection{L\"{o}sungsidee}

Es soll ein binärer Suchbaum implementiert werden.

Dieser soll folgende Funktionen aufweisen:

\begin{enumerate}
	\item bool Contains (TTreeNode const * const pRoot, int const Data);\label{contains}
	\item void Delete (TTreeNode * \& pRoot, int const Data);\label{delete}
	\item void Flush (TTreeNode * \& pRoot);\label{flush}
	\item int Height (TTreeNode const * const pRoot);\label{height}
	\item void InsertSorted (TTreeNode * \& pRoot, int const Data); \label{insertsorted}
	\item void PrintInOrder (TTreeNode const * const pRoot);\label{inorder}
	\item void PrintPostOrder (TTreeNode const * const pRoot);\label{posrtorder}
	\item void PrintPreOrder (TTreeNode const * const pRoot);\label{preorder}
	\item void PrintTree (TTreeNode const * const pRoot, size\_t const Indent);\label{printtree}
\end{enumerate}

ad \ref{contains}:
Die Contains Methode sucht rekursiv ob in dem gegebenen Tree der gesuchte Wert enthalten ist.

ad \ref{delete}:
Die Delete Methode löscht einen Knoten mit dem gegebenen Wert aus dem Baum und hängt die anderen Knoten wieder richtig ein.

ad \ref{flush}:
Flus löscht den gesamten Baum und gibt sämtlichen reservierten Speicher wieder frei.

ad \ref{height}:
Gibt die Höhe des Baumes aus.

ad \ref{insertsorted}:
Fügt dem Baum einen neuen Knoten hinzu, dieser wird auch gleich an der richtigen Stelle einsortiert.

ad \ref{inorder}:
Inorder gibt den Baum in der richtigen Reihenfolge, also vom kleinsten zum Größten Wert.

ad \ref{posrtorder}:
Gibt die Wurzel nach den Teilbäumen aus.

ad \ref{preorder}:
Gibt die Wurzel vor den Teilbäumen  aus.

ad \ref{printtree}
Gibt den Baum von oben nach unten aus, genauso wie er aufgezeichnet werden würde.


Weiters werden noch folgende Hilfsmethoden verwendet:

\begin{enumerate}
	\item TTreeNode * MakeNode (int const Data);\label{makenode}
	\item void PrintNodeData(TTreeNode const * const pNode);\label{prindnodedata}
\end{enumerate}

ad \ref{makenode}:
Mit dieser Methode wird ein neuer Knoten am Heap angelegt.

ad \ref{prindnodedata}
Diese Methode gibt das Data Element der TTreeNode Struktur auf der std::out aus.

\newpage
\subsection{Programmcode in C++}

\subsubsection{Modul Header}

\lstinputlisting[breaklines=true, tabsize=4, showstringspaces=false,backgroundcolor=\color{gray}, numbers=left]
{BinarySearchTree/BinaryTree.h}


\subsubsection{Modul Cpp}

\lstinputlisting[breaklines=true, tabsize=4, showstringspaces=false,backgroundcolor=\color{gray}, numbers=left]
{BinarySearchTree/BinaryTree.cpp}

\newpage
\subsubsection{Testreiber}

\lstinputlisting[breaklines=true, tabsize=4, showstringspaces=false,backgroundcolor=\color{gray}, numbers=left]
{BinarySearchTree/main.cpp}

\newpage
\subsection{Testf\"{a}lle}

Ausgabe des Testtreibers:

\begin{lstlisting}

Tree contains a 3 yes
Tree contains a 10 no
The height of the tree is 2
Pre Order
(5) (3) (2) (4) (7) (6) (8)

In Order
(2) (3) (4) (5) (6) (7) (8)

Post Order
(2) (4) (3) (6) (8) (7) (5)

Print Tree
         (8)
      (7)
         (6)
   (5)
         (4)
      (3)
         (2)


Delete and print Tree
(2) (4) (5) (6) (7) (8)

Flush and print Tree


\end{lstlisting}

\section{Aufgabe 2}

\subsection{L\"{o}sungsidee}

Es soll ein Parser für Arithmetische Ausdrücke gebaut werden.
Dazu wird mittels der EBNF Notation eine Grammatik definiert.
Als Input wird mithilfe der vorhanden Scanner Klasse eine Datei gelesen und aus dieser dann der Arithmetische Ausdruck geparst
und das Ergebnis auf std::out angezeigt. Fehlermeldung werden ebenfalls über std::out angegben.

\newpage
\subsection{Programmcode in C++}

\subsubsection{Modul Header}

\lstinputlisting[breaklines=true, tabsize=4, showstringspaces=false,backgroundcolor=\color{gray}, numbers=left]
{ArithExpr/ArithExprModule.h}


\subsubsection{Modul Cpp}

\lstinputlisting[breaklines=true, tabsize=4, showstringspaces=false,backgroundcolor=\color{gray}, numbers=left]
{ArithExpr/ArithExprModule.cpp}

\newpage
\subsubsection{Testreiber}

\lstinputlisting[breaklines=true, tabsize=4, showstringspaces=false,backgroundcolor=\color{gray}, numbers=left]
{ArithExpr/ArithExpr.cpp}

\newpage
\subsection{Testf\"{a}lle}

Input:
\\
8 * 2 + 4 * 2\\
(10 * 10) - 1\\
5 * 5 + (*1)\\
2 * 3 + 4 / 0
\\\\
Output:
\\
24\\
99\\
Es wurde ein ungueltiges Zeichen in der Eingabe erkannt. Ergebnis wird trotz Fehlers ausgegeben: Es wurde ein weiterer Ausdruck oder eine Zahl erwartet! 25\\
Es wurde ein ungueltiges Zeichen in der Eingabe erkannt. Ergebnis wird trotz Fehlers ausgegeben: Es wurde ein weiterer Ausdruck oder eine Zahl erwartet! 0\\
Es wurde ein ungueltiges Zeichen in der Eingabe erkannt. Ergebnis wird trotz Fehlers ausgegeben: Division durch 0 ist nicht erlaubt 10



\end{document}